\documentclass[11pt]{article}
\usepackage{geometry}\geometry{left=2.5cm,right=2.5cm,top=3.5cm,bottom=3.5cm}
\pagestyle{headings}
\usepackage{enumerate} 
\usepackage{amssymb} \linespread{1} \hyphenation{}
\usepackage{booktabs}
\author{writecoffee} \title{6.6}
%\usepackage{fancyhdr} \pagestyle{fancy} \fancyfoot{} \fancyhead[LE,RO]{\textsc \leftmark~~~~\thepage} \fancyhead[LO,RE]{\textsc \leftmark~~~~\thepage} \renewcommand{\headrulewidth}{0.4pt} 
\begin{document}
\maketitle

{\setlength{\baselineskip}{1\baselineskip}
\setlength{\parindent}{0pt}
\setlength{\parskip}{2ex plus 0.5ex minus 0.2ex}
\begin{enumerate}[1.]
\item
	First Glimpse
	\begin{itemize}
	\item
	The only way a locker could be left open is when it's toggled an odd number of times. So our objective turns to finding those numbers.
	\end{itemize}
\item
	When would a number could be toggled odd times?
	\begin{itemize}
	\item
	If we pair $n$'s factors by their complements, through citing several examples, we can observe that only perfect numbers that 
	consist of odd numbers of factors, e.g., $36 = (1, 36) = (2, 18) = (3, 12) = (4, 9) = (6, 6)$, since $(6, 6)$ only contributes 1 %
	factor, totally 9 (odd) numbers of factors are taken into consideration.
	\item
	There are 10 perfect squares, they are \{1, 4, 9, 16, 25, 36, 49, 64, 81, 100\}, and hence 10 lockers are left open.
	\end{itemize}
\end{enumerate}
\par}
%\bibliographystyle{plain}
%\bibliography{6.6}
\end{document}
